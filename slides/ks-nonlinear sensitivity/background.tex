\section{Background}

\begin{frame}
In this talk, we shall consider the radially symmetric solutions of the chemotaxis system
\begin{equation}\nonumber
	\left\{ \begin{array}{ll}
		u_t = \Delta u - \nabla \cdot(u(1+u)^{\alpha-1} \nabla v),&  t>0, x\in\Omega,\\
		0 =  \Delta v - \mu + u,& t>0, x\in\Omega,	\\
		\partial_\nu u = \partial_\nu v = 0 , & t >0, x\in\partial\Omega,\\
		u(\cdot, 0) = u_0(\cdot), & x\in\Omega,\\
	\end{array}
	\right.
\end{equation}
\begin{itemize}
    \item $\Omega = B_R:=\{x\in\mathbb{R}^n:|x|<R\}\subset\mathbb{R}^n$, $n\geq2$,
    \item $\mu = \mu(t) = \fint_\Omega u(x, t)\dd x$ and 
    \item $\alpha\geqslant\frac2n$.
\end{itemize}

\end{frame}

%%%%%%%%%%%%%%%%%%%%%%%%%%%%%%%%%%%%%%%%%%%%%%%%%%%%%%%%%%%%%%%%%%%%%%

\begin{frame}{Background}
\blue{Keller, Segel (1970) JTB}; \blue{Painter, Hillen (2002) CAMQ}
\begin{equation}\label{sys: general ks}
	\begin{cases}
		u_t = \nabla\cdot(D_u(u,v)\nabla u - S(u,v)\nabla v), & (x, t)\in \Omega\times(0, T),\\
		v_t = D_v\Delta v  - G(v)v + uH(v), & (x, t)\in \Omega\times(0, T),\\
	\end{cases}
\end{equation}
\begin{itemize}
    \item $u = u(x, t)$ denotes the density of cells and 
    \item $v = v(x, t)$ represents the concentration of chemical substance.
\end{itemize}
\end{frame}

\begin{frame}
\blue{J\"ager and Luckhaus (1992) TAMS} studied the following parabolic-elliptic system in a planar domain   
\begin{equation}\label{sys: JL}
\begin{cases}\tag{JL}
u_t = \Delta u - \nabla\cdot(u\nabla v), & x\in\Omega, t > 0,\\
0 = \Delta v - \mu + u, & x\in\Omega, t > 0,\\
\end{cases}
\end{equation}
\begin{itemize}
    \item there exists radial initial function with large mass such that the corresponding solution of \eqref{sys: JL} blows up at finite time, 
    \item while that initial mass is less than a number implies that \eqref{sys: JL} exists a global solution.
\end{itemize}
\end{frame}

\begin{frame}
      \blue{Nagai (1995) AMSA} considered radial solutions of
 \begin{equation}
 \label{sys: nagai pe}
 \begin{cases}\tag{N}
 u_t = \nabla\cdot(\nabla u - u\nabla v), & x\in\Omega, t > 0,\\
 0 = \Delta v - v + u, & x\in\Omega, t > 0, \\
 \end{cases}
 \end{equation} 
 and deduced the two-dimensional critical mass $8\pi$ for blowup.
 \begin{itemize}
    \item if $n=1$ or $n=2$ and $m<8\pi$, then the solution is globally bounded, 
    \item but either $n\geq3$ or $n=2$ and $m>8\pi$, if $\int_\Omega u_0|x|^n\dd x$ is sufficiently small, then $u$ blows up in finite time.
    \item either $q\in\Omega$ and $m > 8\pi$ or $q\in\partial\Omega$ and $m > 4\pi$, if $\int_\Omega u_0|x-q|^2\dd x$ is sufficiently small, then the solution blows up in finite time. \blue{Nagai (2001) JIA}
 \end{itemize}
\end{frame}

%%%%%%%%%%%%%%%%%%%%%%%%%%%%%%%%%%%%%%%%%%%%%%%%%%%%%%%%%%%%%%%%%%%%%%%%%%%%%%%%%%%%%%%%%%%%%%%%%%%%%

\begin{frame}
\frametitle{A novel critical mass phenomenon}
\blue{Winkler (2019) MA} revealed a new critical mass phenomenon that exists in both 2 and higher dimensions. Let $n\geqslant 2$, $R>0$ and $\Omega=B_R$. Denote
\begin{align*}
	S:=\biggl\{u_0\geqslant0 : &u_0\in C^0(\overline{\Omega}) \text{ is radially symmetric, } \\ &\fint_{B_r}u_0\gneqq \fint_{\Omega}u_0 \text{ for all } r\in(0, R).\biggr\}
\end{align*}
there exists a constant $m_c(n, R) > 0$ such that, 
\begin{itemize}
    \item if $m>m_c$, then for \alert{all}
$u_0\in S$
with $\int_\Omega u_0 = m$, 
 \eqref{sys: JL} admits a solution blowing up in finite time; 
    \item but if $m<m_c$, one can find $u_0\in S$ with $\int_\Omega u_0 = m$ such that the corresponding solution of \eqref{sys: JL} is globally bounded.
 \end{itemize}
\end{frame}

\begin{frame}
\begin{equation}\label{H: u_0 for JL}
	\begin{aligned}
	&0\leq u_0\in C^0(\overline{\Omega}),\text{ radially symmetric,}\\
	&\text{\alert{nonincreasing} and  } u_0\not\equiv \text{ const.}
	\end{aligned}\tag{$\mathcal{U}_0$}
\end{equation}

\begin{theorem}[\blue{Souplet, Winkler (2019) CMP}]
Let $\Omega=B_R\subset\mathbb{R}^n$, with $R>0$ and $n\geq2$, and let $u_0$ satisfy \eqref{H: u_0 for JL}. Then there exists $C(\lambda_1,n) > 0$ such that if $\|u_0\|_1 > C(\lambda_1, n) R^{n-2}$, then the solution of \eqref{sys: JL} blows up in finite time. Here, $\lambda_1 > 0$ is the first eigenvalue of $-\Delta$ over $B_1\subset\mathbb{R}^{n+2}$ with homogenerous Dirichlet boundary. 
\end{theorem}
\end{frame}

\begin{frame}{A double critical mass phenomenon}
\noindent \blue{Fuhrmann, Lankeit and Winkler (2022) JMPA}:
\begin{align*}
\begin{cases}\label{sys:N}\tag{N}
u_t=\Delta u-\chi\nabla \cdot\left( u \nabla v\right), &x\in\Omega, t>0,\\
0= \Delta v- kv+u,&x\in\Omega, t>0\\
\frac{\partial u}{\partial \nu}-u\frac{\partial v}{\partial \nu}=v=0, & x \in \partial \Omega, t>0, \\ u(x, 0)=u_0(x)\in C^0(\overline{\Omega}), & x \in \Omega,
\end{cases}
\end{align*}
with $k\geqslant0$, where $\Omega \subset \mathbb{R}^n$ ($n\geqslant2$) is a bounded star-shaped domain with smooth boundary in the sense that
$$
\gamma:=\inf _{x \in \partial \Omega} x \cdot \nu(x)>0 .
$$

There exists $m^\ast(\Omega,k)>0$ such that
\begin{itemize}
    \item  for \alert{all} initial data with $\int_\Omega u_0 > m^\ast$, \eqref{sys:N} admits a solution blowing up in finite/infinite time;
    \item  there exists initial data with $\int_{\Omega} u_0 < m^\ast$ such that the solution of \eqref{sys:N} exists globally and remains bounded.
\end{itemize}
\end{frame}

\begin{frame}
\begin{exampleblock}{Example}
let $n=2$, then $m_\star=8\pi$ is the classical critical value for blowup of both the chemotaxis system \eqref{sys: JL} and \eqref{sys:N}. For brevity, $m^\star\geqslant8\pi$ denotes the above two novel critical mass. $m_\star$ and $m^\star$ separate three cases.
\end{exampleblock}
\begin{figure}
\centering
\includegraphics[width=0.7\linewidth]{double}
%\caption{}
\label{fig:double}
\end{figure}
  
\end{frame}

%\begin{frame}
%Later, \blue{Fuhrmann, Lankeit and Winkler (2022) JMPA} found the system \eqref{sys: nagai pe} subjected to a mixed boundary condition
%has a double critical mass level. These two thresholds separate three ranges: 
%\begin{itemize}
%    \item all solutions are globally bounded, 
%    \item both bounded and exploding solutions exist or 
%    \item all nontrivial solutions blow up.
%\end{itemize}
%\end{frame}

\begin{frame}{Motivation}
Both \eqref{sys: JL} and \eqref{sys:N} are critical in a planar domain, supercritical in higher dimensions. We consider 
\begin{equation}\label{sys: ks-nonlinear-sensitivity}
	\left\{
	\begin{array}{ll}
		u_t = \Delta u - \nabla\cdot (u(1+u)^{\alpha - 1}\nabla v), & x\in\Omega, t > 0,\\
		0 = \Delta v - \fint_\Omega u_0 + u, & x\in\Omega, t > 0,\\
		\partial_\nu u = \partial_\nu v = 0, & x\in\partial\Omega, t > 0,\\
		u(\cdot, 0) = u_0(\cdot)\geqslant0, &x\in\Omega,\\
	\end{array}
	\right.
	\tag{$\bigstar$}
\end{equation}
with supercritical sensitivity $\alpha\geqslant\frac2n(n\geqslant2)$.
\blue{Winkler, Djie (2010) NA}
\begin{itemize}
    \item If $\alpha<\frac2n$, all solutions are globally bounded;
    \item If $\alpha>\frac2n$, there exists the solution blowing up in finite time. 
\end{itemize}

\blue{Tao and Winkler (2012) JDE};

\blue{Cie\'slak and Stinner (2015) JDE}; 



\blue{Winkler (2022) PLMS}
\end{frame}

%\begin{frame}
% proved that $\alpha=\frac2n$ is a critical exponent of the system \eqref{sys: ks-nonlinear-sensitivity} in the following sense:
%\begin{itemize}
%    \item if $\alpha < \frac2n$, all solutions exist globally and remain bounded,
%    \item if $\alpha > \frac2n$, there exist solutions blowing up in finite time.
%\end{itemize}
%$D_u = (1+u)^{\delta-1}$, $S=u(1+u)^{\sigma - 1}$ and $D_v=G=H=1$. Let $\alpha = \sigma + 1 - \delta$.
%\begin{itemize}
%    \item the parabolic-parabolic system \eqref{sys: general ks} has unbounded solutions if $\alpha > \frac2n$,  \blue{Horstmann, Winkler (2005) JDE}; \blue{Winkler (2010) MMAS}; \blue{Cie\'{s}lak, Stinner (2015) JDE}; 
%    \item all solutions of the system \eqref{sys: general ks} exist globally and remain bounded in the case of $\alpha < \frac2n$. \blue{Tao, Winkler (2012) JDE}
%    \item see \blue{Winkler (2022) PLMS} and \blue{Cao, Gao (2023) JDE} for results on the critical case $\alpha=\frac2n$. 
%\end{itemize}
%\end{frame} 