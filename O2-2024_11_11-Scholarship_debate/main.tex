%% This Beamer template is based on the one found here: https://github.com/sanhacheong/stanford-beamer-presentation, and edited to be used for Stanford ARM Lab

\documentclass[10pt,xcolor=dvipsnames]{beamer}
%\mode<presentation>{}
% 如需要中文, 请使用XeLatex
\usepackage{media9}
\usepackage{amssymb,amsmath,amsthm,enumerate}
\usepackage[slantfont,boldfont]{xeCJK}
%\usepackage[utf8]{inputenc}
\usepackage{array}
\usepackage[parfill]{parskip}
\usepackage{graphicx,animate}
\usepackage{caption}
\usepackage{subcaption}
\usepackage{bm}
\usepackage{amsfonts,amscd}
\usepackage[]{units}
\usepackage{listings}
\usepackage{multicol}
\usepackage{multirow}
\usepackage{tcolorbox}
\usepackage{physics}
%\usepackage{movie15}

\usepackage{tikzducks}
\usepackage{soul}
% Enable colored hyperlinks
\hypersetup{colorlinks=true}

% The following three lines are for crossmarks & checkmarks
\usepackage{pifont}% http://ctan.org/pkg/pifont
\newcommand{\cmark}{\ding{51}}%
\newcommand{\xmark}{\ding{55}}%

% Numbered captions of tables, pictures, etc.
\setbeamertemplate{caption}[numbered]
\usepackage{media9} 
%\usepackage[superscript,biblabel]{cite}
\usepackage{algorithm2e}
\renewcommand{\thealgocf}{}

% Bibliography settings
\usepackage[style=ieee]{biblatex}
\setbeamertemplate{bibliography item}{\insertbiblabel}

\addbibresource{bibliography.bib}

% Glossary entries
\usepackage[acronym]{glossaries}
\newacronym{ML}{ML}{machine learning}
\newacronym{HRI}{HRI}{human-robot interactions}
\newacronym{RNN}{RNN}{Recurrent Neural Network}
\newacronym{LSTM}{LSTM}{Long Short-Term Memory}


\theoremstyle{remark}
\newtheorem*{remark}{Remark}
\theoremstyle{definition}

\newcommand{\empy}[1]{{\color{darkorange}\emph{#1}}}
\newcommand{\empr}[1]{{\color{cardinalred}\emph{#1}}}
\newcommand{\examplebox}[2]{
\begin{tcolorbox}[colframe=darkcardinal,colback=boxgray,title=#1]
#2
\end{tcolorbox}}

\usetheme{Stanford} 
\input{./style_files_stanford/my_beamer_defs.sty}
\logo{\includegraphics[height=0.39in]{style_files_stanford/logoseu.jpg}}

\makeatletter
\let\@@magyar@captionfix\relax
\makeatother

\title[一等学业奖学金答辩]{博士一等学业奖学金答辩}


\begin{document}

\author[毛宣]{
	\begin{tabular}{c} 
	\Large
	毛宣 \footnotesize 2021级春博\\ 
    \footnotesize 导师: 李玉祥教授
    %\footnotesize \href{mailto:author.1@seu.edu.cn}{author.1@seu.edu.cn} \\
    % Author Two \\
    % \footnotesize \href{mailto:author.2@kaust.edu.sa}{author.2@kaust.edu.sa}
\end{tabular}
\vspace{4ex}}

\institute{
	\vskip 5pt
	\begin{figure}
		\centering
		\begin{subfigure}[t]{0.5\textwidth}
			\centering
			\includegraphics[height=0.5in]{style_files_stanford/logowhitebg.png}
		\end{subfigure}%
% 		~ 
% 		\begin{subfigure}[t]{0.5\textwidth}
% 			\centering
% 			\includegraphics[height=0.5in]{style_files_stanford/logostatswhbg.png}
% 		\end{subfigure}
	\end{figure}
	\vskip 10pt
	东南大学数学学院\\
	\vskip 3pt
}

% \date{June 15, 2020}
\date{2024/11/11}

\begin{noheadline}
\begin{frame} \maketitle \end{frame}
\end{noheadline}

% \begin{frame}
% 	\frametitle{Overview} % Table of contents slide, comment this block out to remove it
% 	\tableofcontents % Throughout your presentation, if you choose to use \section{} and \subsection{} commands, these will automatically be printed on this slide as an overview of your presentation
% \end{frame}

%\section{Introduction}
% `[allowframebreaks]` can be used to have multiple slides in one frame, where the slides are continued with the suffix "(cont.)"; `[allowframebreaks]` can be used with `\framebreak` to manually break each frame into multiple slides
\begin{frame}[allowframebreaks]
\frametitle{科研成果}
2023年度学术论文\\
\textbf{Xuan Mao}, Yuxiang Li.
Critical Mass for Keller-Segel Systems with Supercritical Nonlinear Sensitivity. 
\emph{Math. Models Methods Appl. Sci.}, 2023, 33(11): 2395--2423.
%\url{https://doi.org/10.1142/S0218202523400079}
%\href{https://doi.org/10.1142/S0218202523400079}{doi:10.1142/S0218202523400079}
\hfill (数学会T1; 中科院1区; JCR: Q1)\\
\textbf{Xuan Mao}, Yuxiang Li. 
Dirac-type aggregation with full mass in a chemotaxis model. 
\emph{Discrete Contin. Dyn. Syst. Ser. S.}, 2024, 17(4):1513--1528. 
%\url{https://doi.org/10.3934/dcdss.2023185}
%\href{https://doi.org/10.3934/dcdss.2023185}{doi:10.3934/dcdss.2023185}
\hfill (数学会T4; 中科院4区; JCR: Q2)\\
近期科研成果和已完成工作(不算作工作量)\\
\textbf{Xuan Mao}, Yuxiang Li. 
Instability of homogeneous steady states in chemotaxis systems with flux limitation, 
\emph{Nonlinear Anal.}, 2024, 243: Paper No. 113527, 18.
    %\url{https://doi.org/10.1016/j.na.2024.113527}
\hfill (数学会T3; 中科院2区; JCR: Q1)\\
\textbf{Xuan Mao}, Yuxiang Li. 
A note on the $8\pi$ problem of J\"ager-Luckhaus system.
%\href{https://doi.org/10.48550/arXiv.2405.06315}
{arXiv: 2405.06315}.\\ 
\textbf{Xuan Mao}, Yuxiang Li.
Global solvability and unboundedness in a fully parabolic quasilinear chemotaxis model with indirect signal production.
%arXiv preprint, 2024, 
%\href{https://arxiv.org/abs/2410.13238}
{arXiv: 2410.13238}.

%\href{https://doi.org/10.1016/j.na.2024.113527}{doi:10.1016/j.na.2024.113527}\\
\end{frame}

\begin{frame}[plain]

    \begin{center}
        \begin{minipage}{1\textwidth}
            \setbeamercolor{mybox}{fg=white, bg=black!60!green}
            \begin{beamercolorbox}[wd=0.70\textwidth, rounded=true, shadow=true]{mybox}
            \LARGE \centering 感谢倾听
            \end{beamercolorbox}
        \end{minipage}
    \end{center}

    % \begin{figure}[!t]
    %     \centering
    %     \includegraphics[width=.8\textwidth]{figures/figure5.png}
    %     \label{figure4_ad}
    % \end{figure}
\end{frame}

% \begin{frame}[plain]
%     \centering
%     \begin{tikzpicture}[scale=2]
%     \duck[speech={\color{Brown} 感谢倾听!}]
%     \end{tikzpicture}

% \end{frame}

% % This demonstrates a new section
% \section{Examples}
% % This demonstrates a single frame without framebreaks
% \begin{frame}{Example of Horizontal Subfigures}

% 	\begin{figure}
% 		\centering
% 		\begin{subfigure}[t]{0.5\textwidth}
% 			\centering
% 			\includegraphics[width=0.9\textwidth]{images/stone2014fall_setup.png}
% 			\caption{Single Kinect setup for fall prevention in elderly residence \cite{stone2014fall}}
% 		\end{subfigure}%
% 		~ 
% 		\begin{subfigure}[t]{0.5\textwidth}
% 			\centering
% 			\includegraphics[width=\textwidth]{images/staranowicz2015easy_multiple_kinects.png}
% 			\caption{Multiple Kinects calibration for fall prediction\cite{staranowicz2015easy}}
% 		\end{subfigure}
% 		\caption{Examples of Horizontal Subfigures}
% 	\end{figure}
% \end{frame}

% \begin{frame}{Example of Horizontal Alignment}
%     % For data collection:
    
%     Example of Horizontal Alignment of a \texttt{table} and a \texttt{figure}.
%     \begin{center}
%     \begin{minipage}[t]{.65\linewidth}
%     \begin{table}[H]
%     % \renewcommand{\arraystretch}{1.3}
%     \caption{Environment limitations on data collection}
%     \label{tab:env_limit}
%     \centering
%     % \begin{tabular}{m{1.6cm}|c|>{\centering\arraybackslash}m{2cm}|>{\centering\arraybackslash}m{2.3cm}}
%     \begin{tabular}{m{2cm}|c|c|>{\centering\arraybackslash}m{1.5cm}}
%     % \begin{tabular}{c|c|c|c}
%         & Kinect & Stereo & Kinect + Stereo\\
%         \hline
%         Indoor & \cmark & \cmark & \cmark \\
%         \hline
%         Outdoor & \xmark & \cmark & \cmark \\
%         \hline
%         High number of features & \cmark & \cmark & \cmark \\
%         \hline
%         Low number of features & \cmark & \xmark & \cmark 
%     \end{tabular}
%     \end{table}
%     \end{minipage}%
%     \begin{minipage}[t]{.35\linewidth}
%     \vspace{0pt}
%     \centering
%     \includegraphics[width=0.7\textwidth]{images/waist_cam_setup_new.png}
%     \end{minipage}
%     \end{center}
% \end{frame}

% \begin{frame}[allowframebreaks]
% \frametitle{Example of resizable equations}

% \begin{center}
% \scalebox{1.0}{\parbox{\linewidth}{%
% 		\begin{align*}
% 		& {\text{min \hskip 6pt}}
% 		& & J = \int (a_{real} - \hat{a})^2  \\
% 		& \text{subject to}
% 		& & \text{human kinematics} \\
% 		&&& \text{no collision} \\
% 		&&& \text{no falling} 
% 		\end{align*}
% }}
% \end{center}
% \end{frame}

% \begin{frame}[allowframebreaks]
% \frametitle{Example of Regular Equations}
%     % \begin{equation}
%     %     {}^Ag = {}^AR_B {}^Bg
%     % \end{equation}
    
%     % \begin{equation}
%     %     V = \frac{{}^Bg \cross {}^Ag}{\norm{{}^Ag}\norm{{}^Bg}}, 
%     %     \theta = \arccos{\frac{{}^Bg \cross {}^Ag}{\norm{{}^Ag}\norm{{}^Bg}}}
%     % \end{equation}
    
%     \begin{equation}
%         \begin{split}
%         {}^AR_{B}(t_0)=\left[\begin{array}{ccc}
%         1 & 0 & 0 \\
%         0 & 1 & 0 \\
%         0 & 0 & 0
%         \end{array}\right]+
%         \sin (\theta)\left[\begin{array}{ccc}
%         0 & -v_{3} & v_{2} \\
%         v_{3} & 0 & -y_{1} \\
%         -v_{2} & v_{1} & 0
%         \end{array}\right]+ \\
%         (1-\cos (\theta))\left[\begin{array}{ccc}
%         0 & -v_{3} & v_{2} \\
%         v_{3} & 0 & -v_{1} \\
%         -v_{2} & v_{1} & 0
%         \end{array}\right]^{2}
%         \end{split}
%         \end{equation}
        
%         \begin{align}
%             {}^AR_{B}(t) &= \Delta R {}^AR_{B}(t_0) \\
%             \Delta R &= {}^AR_{B}(t) {}^AR_{B}^T(t_0)
%         \end{align}
% \end{frame}

% \begin{frame}[allowframebreaks]
% \frametitle{Example of Video}

% 	\includemedia[
% 	width=\linewidth,
% 	totalheight=0.6\linewidth,
% 	activate=pageopen,
% 	passcontext,  %show VPlayer's right-click menu
% 	addresource=images/opensim_video.mp4,
% 	flashvars={
% 		%important: same path as in `addresource'
% 		source=images/opensim_video.mp4
% 	}
% 	]{\fbox{Click!}}{VPlayer.swf}
    
% \end{frame}

% \begin{frame}[allowframebreaks]
% \frametitle{Bibliography}
% \printbibliography
% \end{frame}

\end{document}