% !TEX program = xelatex
% !BIB program = bibtex8


% =============================================================================
%
% IMPORTANT WARNING:
% ==================
%
% You can use this sample as a template for your documents...
%
% However, you should keep your documents in a safe place where they belong.
% WinEdt's Local Application Data Folder will be removed if you uninstall
% the program and ALL documents inside this (%b) folder will be lost!!!
%
% =============================================================================
%--------------PREAMBLE------------------------------------------------------


\documentclass[10pt]{beamer}


\usepackage[english]{babel}
\usepackage{fancyhdr}        % header footer
\usepackage{graphicx}        % figure
%\usepackage{algorithm2e}
%\usepackage{booktabs}
\usepackage{xcolor}
%\usepackage{bookmark}
%\usepackage{stix} 
\usepackage{esint}
%\usepackage{unicode-math}
\usepackage[scheme=plain]{ctex}

%\usepackage{beamerthemesplit}
\usepackage{amsmath}
\usepackage{amsfonts}
\usepackage{amsthm}
%\usepackage{graphicx}
%\usepackage{hyperref}
% you may include additional packages should you need them

%\usetheme{Madrid}
%\usetheme{Xiaoshan}
\usetheme[block=fill,
    sectionpage=none,   %optional: none, simple, progressbar
    progressbar=head,   %optional: none, head, frametitle, foot
    numbering=fraction,   %optional: none, counter, fraction
    ]{metropolis}
%\metroset{block=fill}
%\usefonttheme{serif,professionalfonts}
%\setbeamerfont*{frametitle}{size=\normalsize,series=\bfseries}


\usefonttheme[onlymath]{serif}

\setbeamertemplate{navigation symbols}{}
%\AtBeginSection[]{}

\newcommand{\R}{\mathbb{R}}    % the real numbers
\newcommand{\dd} {\ \mathrm d}
\newcommand{\uwt}{\underline{W}^{(\theta)}}
\newcommand{\uw}{\underline{W}}
\newcommand{\ow}{\overline{W}}
\newcommand{\fna}{\frac{2}{\alpha n}}

\newcommand{\blue}[1]{{\color{blue} #1}}

\newcommand{\weakstar}{
\buildrel\ast\over\rightharpoonup
%\stackrel{\ast}{\rightharpoonup}
%\overset{\ast}{\rightharpoonup}
}

\theoremstyle{remark}
\newtheorem{rem}{Remark}

\title{Critical mass for Keller-Segel systems with supercritical nonlinear sensitivity}

\author{毛宣\inst{1}\and
    导师: 李玉祥\inst{1}}
\institute{
    \inst{1}东南大学数学学院
    }
\date{2023年生物数学趋化模型学术研讨会\\
    电子科技大学, 成都\\
    2023年8月24日
    }
%\date{\zhtoday}

\begin{document}


\begin{frame}
    \usebeamertemplate{title page}
\end{frame}
\frame{\frametitle{Outline}\tableofcontents}


\section{生物背景}

\begin{frame}{生物背景}
  
\begin{itemize}
    \item 生物趋化性是指生命体在化学物质刺激作用下所作出的定向运动.
    %\item 趋化现象于自然界中广泛存在.
    \item 为了研究细胞的\alert{聚集}现象, Keller 和 Segel~\footfullcite{Keller1970}提出了趋化模型
\begin{equation*}\label{sys: general ks ch1}
	\left\{
	\begin{array}{ll}
		u_t = \nabla\cdot(D_u(u,v)\nabla u \alert{- S(u,v)\nabla v}), & (x, t)\in \Omega\times(0, T), \\
		v_t = D_v\Delta v  - G(v)v + uH(v),                   & (x, t)\in \Omega\times(0, T), \\
	\end{array}
	\right.
\end{equation*}
\begin{itemize}
\item $u = u(x, t)$ 表示细胞密度,
\item $v = v(x, t)$ 表示化学信号的聚集度,
\item $D_u(u,v)$ 和 $D_v$ 分别表示细胞和化学信号的扩散强度,
\item $S(u,v)$ 表示趋化灵敏度函数,
\item $-G(v)v + uH(v)$ 表示化学信号的降解和产生.
\item 细胞的运动包括自身的随机扩散和\alert{沿化学信号浓度梯度方向的趋化吸引运动}.
\end{itemize}
\end{itemize}
\end{frame}

%%%%%%%%%%%%%%%%%%%%%%%%%%%%%%%%%%%%%%%%%%%%%%%%%%%%%%%%%%%%%%%%%%%%%%%%%%%%%%%%%%%%%%%%%%%%%%

\section{Our results}
\begin{frame}{Our results}
\begin{equation}\label{h: initial data u0 w0}
\tag{$\mathcal{I}$}
	\begin{cases}
	(u_0, w_0)\in C^0(\overline{\Omega})\times C^1(\overline{\Omega}), \\
	u_0 \text{ and } w_0\text{ are non-negative, radially symmetric, }\\
	\fint_0^ru_0\rho\dd {\rho} \gneqq \int_0^1u_0\rho\dd {\rho}, 
\fint_0^rw_0\rho\dd {\rho} \geqslant \int_0^1w_0\rho\dd {\rho}, r\in(0, 1).
	\end{cases}
\end{equation}

\begin{theorem}[\blue{M-Li-2023-DCDSS}]
	Let $\Omega = B_1$. For arbitrary $(u_0, w_0)$ satisfying \eqref{h: initial data u0 w0} with $m = \int_\Omega u_0 \geqslant32\pi$,  the solution $(u, v, w)$ of \eqref{sys: ks-p-e-o-isp} blows up in infinite time,  i.e., 
	\begin{equation}\nonumber
		\lim_{t\to\infty}\|u(\cdot, t)\|_{L^\infty(\Omega)} = \infty.
	\end{equation}		
	Precisely, 
%	\begin{equation}\label{eq: large time behavior of u}
%	\lim_{t\to\infty}\int_0^ru(\rho, t)\rho\dd {\rho} = \frac{m}{2\pi}\quad\text{for all } r\in(0, 1).
%	\end{equation}
%    or rather,
    as $t\to \infty$
\begin{equation}\label{eq: dirac delta aggregation}
u(\cdot, t)\weakstar m\delta_0(\cdot)
\end{equation}
in $\mathcal{M}(\overline{\Omega})$, the space of measures on $\overline{\Omega}$,
where $\delta_0$ denotes the Dirac distribution with unit weight at zero.
\end{theorem}
\end{frame} 

\begin{frame}

\begin{corollary}[\blue{M-Li-2023-DCDSS}]
	Denote
	\begin{equation}\nonumber
	\begin{split}
	S:= &\biggl\{ M \geqslant 8\pi :
        \eqref{sys: isp} \text{ admits an unbounded solution }\\
    &\quad\text{for all } (u_0, w_0) \text{ satisfying } \eqref{h: initial data u0 w0} \text{ and } M=\int_\Omega u_0\dd{x}\biggr\}.
	\end{split}
	\end{equation}
	Then
	\[m_c := \inf S\]
is well-defined and positive. Moreover, $(m_c,\infty)\subset S$.
\end{corollary}

\begin{block}{Remark}
  We point out that subsolutions  constructed in Section 6 of [Y.Tao and M.Winkler, J. Eur. Math. Soc., 19 (2017), 3641-3678] imply the existence of radial solutions to \eqref{sys: isp} satisfying \eqref{eq: dirac delta aggregation} with $m\in(8\pi,\infty)$, which is not stated explicitly there.
\end{block}

\end{frame}





%%%%%%%%%%%%%%%%%%%%%%%%%%%%%%%%%%%%%%%%%%%%%%%%%%%%%%%%%%%%%%%%%%%%%%%%%%%%%%%%%%%%%%%%%%%%%%%%%

\section{Sketch of proof}
\begin{frame}{Sketch of proof}

\begin{lemma}[\blue{Tao-Winkler-2017-JEMS}]
\label{le: global existence and uniqueness}
	Let $\Omega\subset\mathbb{R}^2$ be a bounded domain with smooth boundary, suppose that $u_0\in C^0(\overline{\Omega})$ and $w_0\in C^1(\overline{\Omega})$ are nonnegative.
    Then there exists a triple $(u, v, w)$ of nonnegative functions solving \eqref{sys: isp} in the classical sense. $(u, v, w)$ can be uniquely identified by the inclusions		
	\begin{align*}
	& u \in C^0(\overline{\Omega} \times[0, \infty)) \cap C^{2,1}(\overline{\Omega} \times(0, \infty)), \\
	& v \in C^{2,0}(\overline{\Omega} \times[0, \infty)), \\
	& w \in C^{0,1}(\overline{\Omega} \times[0, \infty))
	\end{align*}
and the identity
\(\int_{\Omega} v(x, t)\dd x=0\) for all  \(t > 0\).
Furthermore,
\begin{equation*}
	\int_{\Omega} u(x, t) \dd {x}=\int_{\Omega} u_{0} \dd x
    \quad\text{for all } t > 0.
\end{equation*}
In particular, if $\Omega = B_1$, both $u_0$ and $w_0$ are radially symmetric, then $u$, $v$ and $w$ are all radially symmetric.
\end{lemma}
\end{frame}

\begin{frame}{Mass distribution function}
\begin{equation}\label{eq: mass distribution function}
	U(\xi, t):=\int_{0}^{\sqrt\xi} r u(r, t) \mathrm{d} r, \quad \xi \in\left[0, 1\right], t \in\left[0, \infty\right)
\end{equation}
transfers \eqref{sys: isp}  into the Dirichlet problem
\begin{align}\label{sys: partial mass pde}
    \begin{cases}
		\mathcal{P}(U) = 0, & \xi \in(0, 1), t \in\left(0, \infty\right)\\
		U(0, t)=0, \quad U\left(1, t\right)=\frac{m}{2\pi}, & t \in\left(0, \infty\right)\\
		U(\xi, 0) = U_0(\xi), & \xi\in(0, 1), 
    \end{cases}
\end{align}
along with an evident initial condition
\begin{equation}\label{partial mass initial data}
	U_0(\xi) = \int_0^{\sqrt\xi}u_0(r)r\dd r,
\end{equation}
where $m = \int_\Omega u_0(x)$, $W_t(\xi):=\int_0^{\sqrt\xi}w(r, t) r\dd {r}$, $K_t:= W_t(1)$ and
\begin{align*}
\mathcal{P}(U, w_0) &:= U_t - 4 \xi U_{\xi \xi} - 2\left\{\int_0^t e^{-(t-s)}\left(U(\xi, s)-\frac{m\xi}{2 \pi}\right) \dd {s}\right\} \cdot U_{\xi} \\
&\quad - 2(W_0(\xi)-K_0 \xi) \cdot e^{-t} U_{\xi}.
\end{align*}
\end{frame}

\begin{frame}
\begin{lemma}[Comparison method]
	Let $T>0$, $w_0\in C^1(\overline{\Omega})$ and suppose that $\uu$ and $\ou$ are two nonnegative functions which belong to
\[
    C^0([0, 1]\times [0, T])\cap
    C^1((0, 1)\times(0, T))\cap
    C^0((0, T); W^{2,\infty}((0,1)))
\]
and satisfy
\begin{equation*}
		\mathcal{P}(\uu, w_0) \leqslant \mathcal{P}(\ou, w_0)\quad\text{a.e. } \xi\in(0,1) \text{ and for all } t \in(0, T),
\end{equation*}
and if moreover
\(
  0\leqslant \ou_\xi\leqslant L\) for all  \((\xi,t)\in(0,1)\times(0,T)\)
with some $L>0$,
\begin{equation*}
	\uu(0, t) \leqslant \ou(0, t) \quad{and}\quad\uu(1, t) \leqslant \ou(1, t) \quad\text{for all } t \in(0, T),
\end{equation*}
as well as
\begin{equation*}
	\uu(\xi, 0)\leqslant \ou(\xi, 0) \quad\text{for all }  \xi\in (0, 1),
\end{equation*}
then
\begin{equation*}
	\uu (\xi, t) \leqslant \ou (\xi, t) \quad\text{for all } (\xi, t) \in[0, 1]\times [0, T].
\end{equation*}
\end{lemma}
\end{frame}


\begin{frame}
\frametitle{Stationary problems}
The stationary problem of \eqref{sys: isp} can be formulated formally as
\begin{equation}\nonumber
  \begin{cases}
    \Delta u - \nabla\cdot(u\nabla v) = 0, & x\in\Omega, \\
    \Delta v - \fint_\Omega u + u = 0, & x\in\Omega, \\
    w = u & x\in\Omega, \\
    \partial_\nu u = \partial_\nu v = 0, & x\in\partial\Omega,
  \end{cases}
\end{equation}
the component $u$ of which is transformed by the mass distribution function \eqref{eq: mass distribution function} into
\begin{equation}\label{sys: stationary problem of mass distribution pde}
  \begin{cases}
    4\xi \mathscr U_{\xi\xi} + 2\left( \mathscr U-\frac{m\xi}{2\pi}\right)\mathscr U_\xi = 0, & \xi\in(0,1),\\
    \mathscr U(0)=0, \mathscr U(1)=\frac{m}{2\pi}.
  \end{cases}
\end{equation}
\end{frame}



\begin{frame}{Step 1: A continuous family stationary subsolutions}
\begin{lemma}\label{le: stationary subsolution}
	Suppose that $w_0\in C^1(\overline{\Omega})$ is radially symmetric and for some $C_0 > 0$,
	\begin{equation}\label{h: W_0 behaves well}
		W_0(\xi) \geqslant K_0\xi + C_0\xi(1-\xi)\quad\text{for all } \xi\in[0,1].
	\end{equation}
    %where $W_0$ and $K_0$ are defined in \eqref{eq: W_t} and \eqref{eq: K_t}, respectively.
	Then for any choice of $m \geqslant32\pi$, there exists $\theta_0=\theta_0(C_0)\in(0,1)$ with the following property:  One can find families $\{\xi_0^{(\theta)}\}_{\theta\in[0, 1)}\subset(0, 1)$ and $\{\uut\}_{\theta\in[0, 1)}\subset C^1([0, 1])$ such that for all $\theta\in(0, 1)$, $\uut$ moreover belongs to $C^2([0, 1]\backslash \{\xi_0^{(\theta)}\})\cap W^{2, \infty}((0, 1))$ and satisfies $\uut(0) = 0$, $\uut(1) = \frac{m}{2\pi}$ and $\uut_\xi\geqslant 0$ in $(0, 1)$ as well as
	\begin{equation}\label{eq: subsolution equation}
		\mathcal{P}(\uut, w_0) < 0 \quad\text{for all } \xi\in(0, 1)\setminus \{ \xi_0^{(\theta)}\} \text{ and } \theta\in(0, \theta_0),
	\end{equation}
	and such that $[0, 1) \ni \theta \mapsto \uut $ is continuous as a $C^1([0, 1])$-valued mapping, where
	\begin{equation}\label{eq:uu0}
		\underline{\mathscr U}^{(0)}(\xi) = \frac{m\xi}{2\pi} \quad\text{for all } \xi\in(0, 1)
	\end{equation}

\end{lemma}

\end{frame}


\begin{frame}

\begin{block}{Continued}
	and
	\begin{equation}\label{eq:uut pp converges to m/2pi}
		\uut (\xi) \rightarrow \frac{m}{2\pi} \quad\text{for all } \xi\in(0, 1) \text{ as } \theta\nearrow 1.
	\end{equation}
	Particularly, for all  $\xi\in (0, 1)\setminus \{ \xi_0^{(\theta)}\}$ and $\theta\in(0,1)$.
	\begin{equation}\label{eq:stationary subsolution inequality}
		4\xi\uut_{\xi\xi} + 2\uut_\xi\left(\uut -\frac{m\xi}{2\pi}\right) > 0
	\end{equation}
\end{block}

	Fixing $m\geqslant32\pi$, for $\theta\in[0,1)$ and $d=d(\theta)=\frac{m(1-\theta)}{2\pi}\in(0,\frac{m}{2\pi}]$, we denote
	\begin{equation}\label{parameter c b s}
	c = \left(\frac{2\pi d}m\right)^3, \xi_0 = \frac{\pi d}m, b = 4\left(1 - \frac{2\pi d}m\right)
	\end{equation}
	 and
	\begin{equation}\label{parameter a}
		a= \frac{d}{c}(b\xi_0 + c)^2 = \frac{m}{2\pi}\left(2\left(1-\frac{2\pi d}m\right) + \left(\frac{2\pi d}m\right)^2\right)^2.
	\end{equation}
We define
	\begin{equation}\label{eq: subsolution with parameter}
		\uut =
		\begin{cases}
			\frac{a\xi}{b\xi + c} & \xi\in[0, \xi_0),\\
			\frac{m}{2\pi} + d(\xi - 1) & \xi\in[\xi_0, 1].\\
		\end{cases}
	\end{equation}
\end{frame}

\begin{frame}{Step 2: Nonexistence of regular stationary solutions}
\begin{lemma}\label{le: constant stationary solution}
	Let $m\geqslant32\pi$, and suppose that $\mathscr U\in C^2((0, 1])$ is a nonnegative solution of
\begin{equation}
	\label{sys: one-point boundary value problem}
	\begin{cases}
		4\xi \mathscr U_{\xi\xi} + 2\left(\mathscr U - \frac{m\xi}{2\pi}\right)\mathscr U_\xi = 0,& \xi\in(0,1),\\
		\mathscr U(1) = \frac{m}{2\pi},
	\end{cases}
\end{equation}
 such that $\mathscr U_\xi\geqslant 0$ in $(0, 1)$, and that moreover
	\begin{equation*}
		\mathscr U(\xi)\geqslant\frac{m\xi}{2\pi} \quad\text{for all } \xi\in(0, 1),
	\end{equation*}
but
	\begin{equation*}
		\mathscr U(\xi)\not\equiv\frac{m\xi}{2\pi}.
	\end{equation*}
Then
	\begin{equation*}
		\mathscr U(\xi) \equiv \frac m{2\pi}\quad\text{for all } \xi\in(0, 1).
	\end{equation*}
\end{lemma}
\end{frame}

\begin{frame}
	Define $$S:=\{\theta\in[0,1)\mid\uut\leqslant\mathscr U\},$$
	where $\uut$ is defined as in the lemma of \textbf{Step 1}.

	We claim that $S$ is identical with $[0,1)$ by a connectivity argument, which combining with \eqref{eq:uut pp converges to m/2pi}, i.e.,
\begin{equation*}
  		\uut (\xi) \rightarrow \frac{m}{2\pi} \quad\text{for all } \xi\in(0, 1) \text{ as } \theta\nearrow 1.
\end{equation*}
 implies that $\mathscr U$ is coincided with $\frac{m}{2\pi}$.

\end{frame}

\begin{frame}{Step 3: Blowup of solutions emanating from $\uut$}
\begin{lemma}\label{le: subsolution blow up}
	Let $\theta_0\in(0,1)$ and for $m \geqslant32\pi$, the family
\[
    \{\uut\}_{\theta\in(0,1)}\subset C^1([0, 1])\cap W^{2, \infty}((0, 1])
\]
and $w_0$ be as in Lemma~\ref{le: stationary subsolution}.
For $\theta\in (0, \theta_0]$, let
\[
    U^{(\theta)}\in C^0([0, \infty); C^1([0, 1])) \cap C^{2,1}((0, 1]\times(0, \infty))
\]
denote the global solution of \eqref{sys: partial mass pde},
as obtained through Lemma~\ref{le: global existence and uniqueness} (global existence) and the mass distribution function \eqref{eq: mass distribution function} when applied to the initial data given by $u_0(r) := 2\uut_\xi(r^2), r\in [0, 1]$.
	Then for any $\theta\in(0,\theta_0]$
	\begin{equation*}
		U^{(\theta)}(\xi, t) \rightarrow \frac{m}{2\pi}
        \quad\text {for all } \xi \in(0, 1) \text { as } t\to\infty.
	\end{equation*}
\end{lemma}
\end{frame}



\begin{frame}

\begin{enumerate}
  \item 	We first prove
	\begin{equation*}
U_t^{(\theta)}(\xi, t)\geqslant0 \quad\text{in } (0, 1)\times(0, \infty).
	\end{equation*}
Noting $0\leqslant U^{(\theta)} \leqslant \frac{m}{2\pi}$, there exists a bounded and nondecreasing function $\mathscr U^{(\theta)}:(0,1]\mapsto(0,\frac{m}{2\pi}]$ such that $\mathscr U^{(\theta)}\gneqq\frac{m\xi}{2\pi}$ and
\[
    U^{(\theta)}(\xi, t) \nearrow \mathscr U^{(\theta)}(\xi)
    \quad\text{for all } \xi\in(0,1), \text{ as }  t\to\infty.
\]
  \item By parabolic regularity theory and compactness arguments, we have 
      $$U^{(\theta)}(\xi, t) \to \uut \quad\text{ in } C^2_{\mathrm{loc}}((0, 1]), \text{ as }t\to\infty.$$ 
      and
      $$U^{(\theta)}_t(\xi, t) \to 0 \quad\text{ in } C^0_{\mathrm{loc}}((0, 1]), \text{ as }t\to\infty.$$ 
  \item By L'Hospital's rule, we obtain that $\uut$ solves the one-point boundary problem \eqref{sys: one-point boundary value problem}. Thanks to \textbf{Step 2}: nonexistence of regular stationary solutions, we get $\uut\equiv\frac{m}{2\pi}$.
\end{enumerate}
\end{frame}

\begin{frame}{Step 4: Refined initial conditions for aggregation}
\begin{enumerate}
  \item Adopting the argument of Lemma~3.12 in \blue{Winkler-2019-MA}, we deduce from initial conditions \eqref{h: initial data u0 w0} that there exist $C > 0$ and $t_0>0$ such that
      \[
      U(\xi , t_0) \geqslant \frac{m\xi}{2\pi} + 2C\xi(1 - \xi)
\quad\text{for all } \xi\in (0, 1),
\]
and consequently
\(
W_{t_0} - K_{t_0}\xi \geqslant  C_1\xi(1-\xi)\) for some \(C_1>0\).
  \item Let $\tilde{U}:= U(\xi, t + t_0)$, then $\tilde{U}$ solves 
      \[
      \mathcal{P}(\tilde{U}, w(\xi, t_0)) = 0,\quad(\xi, t)\in(0,1)\times(0,\infty).
      \]
      By \textbf{Step 1}, there exists \(\theta_0\in(0,1)\) such that \(\underline{\mathscr U}^{(\theta_0)}\) is a subsolution of $\mathcal{P}(\tilde{U}, w(\xi, t_0)) = 0$.
  \item By \textbf{Step 3}, we have
  \[
    \frac{m}{2\pi} = \lim_{t\to\infty}U^{(\theta_0)}\leqslant\liminf_{t\to\infty}\tilde{U}\leqslant\limsup_{t\to\infty}\tilde{U}\leqslant\frac{m}{2\pi} \quad\text{for all } \xi\in(0,1),
\]
and hence
\[
    \lim_{t\to\infty}U=\frac{m}{2\pi}\quad\text{for all } \xi\in(0, 1).
\]
\end{enumerate}
\end{frame}











%\begin{frame}
%\blue{Strohm, Tyson and Powell (2013) BMB}
%\begin{align}
%\begin{cases}
%	\label{sys: ks-p-e-o-isp}
%		u_t = \Delta u - \nabla \cdot(u\nabla v),&  t>0, x\in\Omega,\\
%		0 =  \Delta v - \mu + \alert{w},&  t>0, x\in\Omega,	\\
%		w_t + w = u, &  t > 0, x\in\Omega, \\
%		\partial_\nu u = \partial_\nu v = 0 , & t >0, x\in\partial\Omega,\\
%		u(\cdot, 0) = u_0, w(\cdot, 0) = w_0, & x\in\Omega
%\end{cases}\tag{ISP}
%\end{align}
%\blue{Tao, Winkler (2017) JEMS}
%\begin{itemize}
%  \item let $\Omega\subset\mathbb{R}^2$, the classical solutions of \eqref{sys: ks-p-e-o-isp} are \alert{global} for all non-negative initial functions $(u_0, w_0)\in C^0(\overline{\Omega})\times C^1(\overline{\Omega})$, 
%  \item let $\Omega=B_1\subset\mathbb{R}^2$, there exists a novel mass threshold for blowup in infinite time: 
%        \begin{itemize}
%            \item for radial initial functions with $\int_\Omega u_0 < 8\pi$, the solutions are bounded; 
%            \item there exist radial initial functions with $\int_\Omega u_0 > 8\pi$ such that the solutions are unbounded.
%        \end{itemize}
%\end{itemize}
%\blue{Lauren\c{c}ot (2019) DCDSB} extended the results above to nonradial cases.
%\end{frame}






\begin{frame}[standout]
    Thanks for your attention.
    
    Comments are welcome.
\end{frame} 
%--------------END-----------------------------------------------------------

\end{document}
