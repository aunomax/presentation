\section{教育经历}

\begin{frame}{教育经历}
	\begin{itemize}
		\item 2014.09--2018--06 东南大学数学学院; 数学与应用数学; 本科理学学士;
		\item 2018.09--2021.03 东南大学数学学院; 数学硕士研究生; 导师李玉祥教授; 硕博连读;
		\item 2021.03--至今, 东南大学数学学院; 数学博士研究生; 导师李玉祥教授; 预计2025.03毕业
	\end{itemize}
\end{frame}

\section{研究方向}

\begin{frame}{研究方向---抛物型偏微分方程;生物趋化模型解的爆破性质}
  
\begin{itemize}
    \item 生物趋化性是指微生物在化学物质刺激作用下所作出的定向运动.
    %\item 趋化现象于自然界中广泛存在.
    \item 为了研究细胞的\alert{聚集}现象, Keller 和 Segel~\footfullcite{Keller1970}提出了趋化模型
\begin{equation*}\label{sys: general ks ch1}
	\left\{
	\begin{array}{ll}
		u_t = \nabla\cdot(D_u(u,v)\nabla u \alert{- S(u,v)\nabla v}), & (x, t)\in \Omega\times(0, T), \\
		v_t = D_v\Delta v  - G(v)v + uH(v),                   & (x, t)\in \Omega\times(0, T), \\
	\end{array}
	\right.
\end{equation*}
\begin{itemize}
\item $u = u(x, t)$ 表示细胞密度,
\item $v = v(x, t)$ 表示化学信号的聚集度,
\item $D_u(u,v)$ 和 $D_v$ 分别表示细胞和化学信号的扩散强度,
\item $S(u,v)$ 表示趋化灵敏度函数,
\item $-G(v)v + uH(v)$ 表示化学信号的降解和产生.
\item 细胞的运动包括自身的随机扩散和\alert{沿化学信号浓度梯度方向的趋化吸引运动}.
\end{itemize}
\end{itemize}
\end{frame}